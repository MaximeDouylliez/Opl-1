
%% \documentclass[final,5p,times,twocolumn]{elsarticle}
\documentclass[paper=a4, fontsize=11pt]{scrartcl}
\usepackage[T1]{fontenc}
\usepackage{fourier}
\usepackage[english]{babel}

\usepackage[protrusion=true,expansion=true]{microtype}
\usepackage{amsmath,amsfonts,amsthm} % Math packages
\usepackage[pdftex]{graphicx}
\usepackage{hyperref}


%%% Custom sectioning
\usepackage{sectsty}
\allsectionsfont{\centering \normalfont\scshape}


%%% Custom headers/footers (fancyhdr package)
\usepackage{fancyhdr}
\pagestyle{fancyplain}
\fancyhead{}% No page header
\fancyfoot[L]{}% Empty 
\fancyfoot[C]{}% Empty
\fancyfoot[R]{\thepage}% Pagenumbering
\renewcommand{\headrulewidth}{0pt}% Remove header underlines
\renewcommand{\footrulewidth}{0pt}% Remove footer underlines
\setlength{\headheight}{12pt}


%%% Equation and float numbering
\numberwithin{equation}{section}% Equationnumbering: section.eq#
\numberwithin{figure}{section}% Figurenumbering: section.fig#
\numberwithin{table}{section}% Tablenumbering: section.tab#


%%% Maketitle metadata
\newcommand{\horrule}[1]{\rule{\linewidth}{#1}} % Horizontal rule

\title{
  % \vspace{-1in} 
  \usefont{OT1}{bch}{b}{n}
  \normalfont \normalsize \textsc{M1 - IAGL | M Monperrus}
  \\ [25pt]
  \horrule{0.5pt} \\[0.4cm]
  \huge Spoon Report\\
  \horrule{2pt} \\[0.5cm]
}
\author{
  \normalfont \normalsize
  Bailleul Quentin, Douylliez Maxime\\[-3pt]\normalsize
        \today
}
\date{}


%%% Begin document
\begin{document}
\maketitle
\tableofcontents
%% Title, authors and addresses

% \title{Spoon Report - }


% \author{Bailleul Quentin, Douylliez Maxime}
\newpage

\section{Introduction}
Nowadays, IDE are intensively used to develop programs.  IDE allows
the developer to design better programs by using a set of useful tools
while staying in the same environment. IDE usually includes
refactoring functions as a tools to speed up code manipulation safely.
\newline


While most IDE users don't use all implemented refactorings, Some
refactorings are valuable enough to an user but not implemented in
mainstream IDE. Moreover, there are some specific refactoring that can
be very useful to a specific project, but it's up to the project Team
to create such a refactoring. What tools a developer can use to create
his own refactoring in Java?
\newline

This project aims at developing a refactoring that converts Java
source code using JDK collection to a source code using Guava
collection.
\newpage


\section{Context}
Spoon has been used to create refactor while Guava is the library that
will replace the JDK one for a small portion of collection part. Both
will be presented in this part.  


\subsection{What is Spoon ?}
Spoon is a framework for the analysis and transformation of Java
source Code. It can be used to write specific analyses and
transformation far more easily than with other tools. Spoon user
writes his domain specific analysis and transformation in Java ,
avoiding to learn another language to process Java source
code. \newline 

Spoon API allows to manipulate an abstract tree syntax containing all
meaningful informations of a Java source code in order to extract
informations, edit, remove, add part of Java code and generate
modified source code.\newline

Spoon project site can be found at \url{http://spoon.gforge.inria.fr/}

\subsection{why guava ?}

Guava is the open-sourced version of Google's core Java libraries.
Libraries let you use easily a set of functions that you don't need to
write and test. 
Guava have been heavily tested and optimized by Google's developers
and is used intensively.\newline

The JDK Collections API, is widely used amongst developers and has
simplified collections use. 
Guava aims to make working in the Java language faster and easier by
creating new classes or working around JDK classes.

\newpage
For instance, The JDK offers to create a unmodifiable collection from
a mutable collection. But it actually returns a view of the original
collection. Change the mutable collection and the unmodifiable
collection will be too. Developers often use unmodifiable collections
as a base to create immutable collections. Guava offers you to create
truly immutable and distinct version of a collection from a mutable
collection or directly by instantiation.\newline


For instance,this is how to create an immutable List with JDK then with Guava.
\begin{verbatim}
List<String> COLOR_NAMES = new ArrayList<String>();
COLOR_NAME.add("red");.....

List<String> COLOR_NAMES_UNMODIFIABLE=Collections.unmodifiableList(
COLORNAME);

//delete all reference to COLOR_NAMES...

Guava
ImmutableSet<String> COLOR_NAMES = ImmutableSet.of(  
"red",  "orange","yellow",  "green",  "blue",  "purple");


\end{verbatim}

Guava pushes interesting part of JDK further in order to make the life
of a java developer smoother.\newline

Guava libraries can be found at \url{https://code.google.com/p/guava-libraries/}

\newpage
\section{Contribution}
\subsection{Boundaries}
Doing the entire refactoring can represent a huge amount of work to be
done smartly because several lines of "JDK code" can be converted into
a single line of "Guava code".\newline

Furthermore, due to the multiple manners (good or esoterics) to write
a process logic,  determine the complete set of code patterns that `
do something' can be unproductive. \newline

That's why refactoring code patterns have to accept a certain level of inaccuracy , discussed in
pattern code detections part. 

\subsection{Refactoring Arraylist and LinkedList Instantiation}

This part introduces the use of Spoon abstract syntax tree by changing
the way ArrayList and LinkedList are instancied. These transformations
are easy and the Guava resulted is of little interest when java 7 or
over is used. Theses functions are quoted as "Not all Guava features
have much utility (see e.g. Lists.newArrayList)",cf
(\url{https://code.google.com/p/guava-libraries/wiki/PhilosophyExplained})

Before Java 7, compiler could not infer of parameterizable classes
instantiation types. for instance:
\begin{verbatim}

List<TypeThatsTooLongForItsOwnGood> list = new
ArrayList<TypeThatsTooLongForItsOwnGood>();
\end{verbatim}

Guava lets you use an easier way by writing:
\begin{verbatim}
List<TypeThatsTooLongForItsOwnGood> list = Lists.newArrayList();
\end{verbatim}

Now by using Java 7 or over you can write:
\begin{verbatim}
List<TypeThatsTooLongForItsOwnGood> list = new ArrayList<>();
\end{verbatim}

We get all CTvariables then we extract CtExpressions of them. if the
CtExpression if an ArrayList or an LinkedList, then we create a new
CTExpression containing the new instanciation, using the following
code:

\begin{verbatim}
CtExpression<LinkedList<java.lang.String>> value =
getFactory().Code().createCodeSnippetExpression(s);
\end{verbatim}

We replace the old expression by the new one and the transformation is
done.\newline

To go further into refactoring, we need to use code patterns analysis,
in the next subsection.

\subsection{Code patterns detections and transformations}

This part focus on code patterns detections and transformations in
order to translate it to a single Guava call. It offers a glimpse of
what Spoon is capable of.\newline

In Java, with JDK, you have to create a List, then you can add
elements inside:
\begin{verbatim}
ArrayList<String>() list = new ArrayList<>();
this.list.add("First");
this.list.add("Second");
this.list.add("Third");
this.list.add("Fourth");
\end{verbatim}

Guava offers a more pleasant way to do this:
\begin{verbatim}
ArrayList<String>() list =
Lists.newArrayList("First","Second","Third","Fourth");
\end{verbatim}

The major issue encountered during analysis phase is that you might
encounter this code to analyse:

\begin{verbatim}
ls = new LinkedList<Integer>();
ls.add(99);
ls.removeFirst();
if(randomVar.isTrue()){
  AnotherRefOfls.destroy();
}
ls.addFirst(45)
ls.add(1)
\end{verbatim}


Check Statement and possible interactions with processed list by
another references of them is another pain.\newline

Check three types of invocation on a same list that might influence
the final order of parameters is quite hard to implement and should
not be encountered in well written classes. \newline




"Well written class" is blur itself, so there is the worst case where
our processor will preserve initial programming logic:

In the constructor only, you can create as many lists as you want to,
then only use as many list.add () as you want with other variables
instructions. Within these other instructions do not interact with
processed lists.\newline

In the \nameref{app:one}, there is an example of most complex Java Source code
successfully processed by our processor.\newline

The limitation "Constructor only" come from different Methods and
Constructor representation  in Spoon metamodel. CtConstructor and
CtMethod have to be processed in two different classes in our
implementation.

\section{Conclusion}

By analysing and transforming Java code source with Spoon, we make the
first move toward a full migration from JDK collections to guava
equivalent. Creating an API to convert from Java to guava would need a
lot of work and intelligence to be really useable on big
project. Google could afford to spend months to create such a
converter, increasing attractiveness of their libraries even more.  

\newpage
\appendix
\section{Appendix}
\label{app:one}
\begin{verbatim}
ls = new ArrayList<String>(); 
fal= new ArrayList<String>();

ls.add("Hello");
anotherVar2.add(1);
fal.add("Love");
System.Out.println("i do not break analysis")
ls.add(Mister);

var.add("hello3");
fal.add("chocolate");
another.add(2);
\end{verbatim}
Should produce:
\begin{verbatim}
ls = Lists.newArrayList("Hello","Mister","Monperrus"); 
fal= Lists.newArrayList("Love","chocolate");

anotherVar2.add(1);
System.Out.println("i do not break analysis")
var.add("hello3");
another.add(2);
\end{verbatim}





\end{document}
